\documentclass[12pt]{article}
%\usepackage{hyperref}
\usepackage{amsmath}
\usepackage{amssymb}
\usepackage{amsthm}
%\usepackage{asymptote}
\usepackage{calc}
%\usepackage{caption}
%\usepackage{chemfig}
\usepackage{color}
\usepackage{commath}
%\usepackage{enumitem}
%\usepackage[c]{esvect}
%\usepackage{etoolbox}
\usepackage{fancyhdr}
%\usepackage{float}
%\usepackage{fontspec}    %fontspec only works with xetex and luatex.
%\usepackage{fp}
\usepackage{geometry}
%\usepackage{graphicx}
\usepackage{lastpage}
%\usepackage{listings}
%\usepackage{luacode}
%\usepackage{makeidx}
\usepackage{mathtools}
%\usepackage{mhchem}
%\usepackage{pgfplots}
%\usepackage{setspace}
%\usepackage{siunitx}
%\usepackage{tcolorbox}
%\usepackage{tikz}
%\usepackage{todonotes}
%\usepackage{pgfmath}
\usepackage{biblatex}

\addbibresource{refs.bib}

%\setmainfont{DejaVu Sans}    %This is part of the fontspec package.
%\doublespacing

% This very important is you don't want code colored in grey scale
%\lstset{
%    basicstyle=\ttfamily, % This is equivalent to monospace
%    numbers=left,
%    numberstyle=\color{magenta},
%    keywordstyle=\color{blue},
%    stringstyle=\color{red},
%    commentstyle=\color{green}\textit
%}

% For multipage align*
%\allowdisplaybreaks


\newcommand{\keyword}[1]{\textbf{#1}\index{#1}}
\newcommand{\keywordidx}[2]{\textbf{#1}\index{#2}}

\newcommand{\ceil}[1]{\left\lceil #1 \right\rceil}
\newcommand{\floor}[1]{\left\lfloor #1 \right\rfloor}
\newcommand{\avg}[1]{\overline{#1}}
\newcommand{\sd}[1]{\text{SD}\del{#1}}
\newcommand{\var}[1]{\text{Var}\del{#1}}

\newcommand{\N}{\mathbb{N}}
\newcommand{\Z}{\mathbb{Z}}
\newcommand{\R}{\mathbb{R}}

\newcommand{\beq}{::=}
\newcommand{\bs}[1]{\mathrm{#1}}
\newcommand{\bstr}[1]{\mathrm{"#1"}}

\newcommand{\newtyvar}{\mathrm{newtyvar}}
\newcommand{\inst}[1]{\mathrm{inst}\del{#1}}
\newcommand{\gen}[2]{\mathrm{gen}\del{#1, #2}}
\newcommand{\freevars}[1]{\mathrm{freevars}\del{#1}}
\newcommand{\unify}[2]{\mathrm{unify}\del{#1, #2}}

\newcommand{\env}[1]{\langle #1 \rangle}

\newcommand{\thmlet}[2]{\mathrm{let}~#1~\mathrm{in}~#2}
\newcommand{\thmfix}{\mathrm{fix}}
\newcommand{\thmcase}[1]{\mathrm{case}\sbr{#1}}

\newtheorem{theorem}{Theorem}
\newtheorem{lemma}{Lemma}


\title{THM: A Hindley-Milner Typed Language}
\author{Taran Lynn}

\begin{document}
\maketitle


\section{Introduction}

In this paper I will cover the design and implementation of the THM
programming language.
THM is a minimalist Hindley-Milner typed programming language with
algebraic data types.
The goal of designing THM was to create a language and implementation
that was simple enough that it could be used as a teaching tool for
courses on type inference.
To this end the it has a small specification (see the appendix) and a
small implementation (less than 1,000 lines of Haskell).

THM is loosely base on the ML family of languages, specifically
Standard ML.
Besides dropping support for modules, other major changes were made to
the syntax and semantics to reduce the size of the specification and
implementation.
For example, unlike most MLs, THM is lazily evaluated.
This is relatively trivial to implement, and avoids the need to
implement other control structures.
Another major change is that there is no \verb~case~ construct.
Instead there is a \verb~case~ \textbf{function}, which acts as the
elimination rule for a type.

The THM interpreter is implemented in Haskell.
To use it the user simply gives it a list of files with type and
function definitions to type check and load, and the interpreter drops
the user into a read/eval/print loop (REPL) that process THM
expressions.
An example of one definition file is
\begin{verbatim}
type Nat [ Z S(Nat) ]
fix natInd x f = case[Nat] x (\p. f (natInd x f p));

pred = case[Nat] Z (\p. p);
fix add m n = case[Nat] n (\p. S (add p n)) m;
fix mult m n = case[Nat] Z (\p. add n (mult p n)) m;
\end{verbatim}
, and a corresponding REPL session is
\begin{verbatim}
\> Z
Z : Nat
\> S
S : Nat -> Nat
\> mult (S (S Z)) (S (S (S Z)))
S (S (S (S (S (S Z))))) : Nat
\>
\end{verbatim}

The following sections of the paper will give an intutionistic
description of the background theory, design, and implementation of
THM.
For a more formal description, please see the appendix.


\section{Background and Related Work}


Lambda calculus was originally developed by Alonzo Church as a
foundational formulation of mathematics and computation.
However, because the lambda calculus allowed for non-reducible
terms, it was shown to be inconsistent.
To get around this Church introduced types to the lambda calculus,
creating the simply typed lambda calculus\cite{Church1940AFO}.
When lambda calculus began to be as a bases programming languages,
computer scientists also started adopting its type
theories\footnote{By this time there where many such type theories.}.

A mathematician by the name of J. Roger Hindley later worked on a
method for deducing types of SKI combinator
expressions\cite{Hindley1969}.
About a decade later Robin Milner worked on ML, the metalanguage for
the LCF proof system, based on typed lambda calculus.
As part of ML there was an algorithm to infer the types of terms to
reduce programmer overhead.
This algorithm W for inferring types was later presented in
\cite{MILNER1978}, which they later noted corresponded to Hindley's
work.
Luis Damas then refined the corresponding type system as part of his
PhD thesis\cite{Damas1982,Damas1984}.
The resulting type system is called the Hindley-Milner-Damas or
Hindley-Milner typing.

It should be noted that Hindley-Milner type inference relies on a
unification algorithm to work.
Unification is a process that take multiple equations relating
constructed objects that contain variables, and determining what
values for the variables could make the equations valid.
My work uses the unification algorithm presented by Martelli and
Montanari\cite{Martelli1982}.
This unification algorithm was chosen because it is fairly
straightforward to implement and is decently efficient.

Several languages later built off of the work of ML.
A notable example is Standard ML (SML)\cite{Milner1997TheDO}, which is
notable among programming languages for having its specification fully
mathematically formalized.
More mainstream languages based off of ML include
Haskell\cite{Marlow2010Haskell2L} and Rust.
One important component of the languages is algebraic data types
(ADTs).
ADTs form a convenient way to describe data, and are important in
Hindley-Milner typed languages because the restrictions necessary for
decidable inference limits the usability of Church encodings.


\section{Design}

\subsection{Base Language}

At the heart of THM is the lambda calculus.
Expressions, denoted by $e$, are formed by several different
constructs.
\begin{description}
\item[Variables] Denoted by $v$, refer to the parameters of
  encapsulating lambda expressions.

\item[Lambda Abstractions] Denoted by $\lambda v. e$, are anonymous
  functions that take $v$ as a parameter and return $e$.
  Lambda abstractions group to the right, so $\lambda u. \lambda v. e$
  is equivalent to $\lambda u. (\lambda v. e)$.
  Note that the user may use $\backslash$ instead of $\lambda$ for
  ease of use on ASCII keyboards.

\item[Applications] Denoted by $e_1~e_2$, apply the term $e_2$ to
  $e_1$.
  Since THM is lazy evaluated, we evaluating applications by first
  evaluating $e_1$.
  If $e_1$ evaluates to $\lambda v. e_3$, then we evaluate $e_3$ with
  $v$ bond to the thunk formed from $e_2$ (see appendix
  \ref{sec:dynamic_sem} for a formal description).
  Additionally application rules also exist, which will be detailed
  in later parts of the text.

\item[Let Bindings] Denoted by $(\thmlet{v = e_1;}{e_2})$ binds the
  thunk formed from $e_1$ to $v$ and evaluates $e_2$.

\item[Fix-point Operator] Denoted by $\thmfix$, has a special
  application rule such that $\thmfix~f$ evaluates to $f~(\thmfix~f)$.
  This allows for recursion in functions.
\end{description}
Beyond these basic forms some syntactic sugar is also added to the
base language (see appendix \ref{sec:concrete_syntax}).


\subsection{Type Inference}

Given any expression, THM will decide if the expression is typeable,
and if so will infer the most general type for the expression.
Types, denoted by $t$, in the base THM have several constructions.
\begin{description}
\item[Type Variables] Denoted by $`\tau$, type variables represent
  unspecified types.

\item[Function Types] Denoted by $t_1 \to t_2$, represent functions
  that take arguments of type $t_1$ and return values of type $t_2$.
  Like lambda abstractions these group to the right, so
  $t_1 \to t_2 \to t_3$ is equivalent to $t_1 \to (t_2 \to t_3)$.

\item[Quantified Types] Denoted by $\forall `\alpha. t$, these denote
  types where the type variable $`\alpha$ can change.
  If one likens type variables to expression variables, then
  quantified types can be likened to lambda abstractions.
  One thing to note is that only let binded variables may have
  quantified types, and not any other expression or value.
  This restriction is necessary for type inference to be decidable.
\end{description}

Before I describe how to infer what the most general type of
an expression, let me briefly describe type unification.
Suppose we have a set of equations between types (i.e.
$t_1 = t_2, t_3 = t_4, \ldots$), the goal of type unification is to
determine what types can be substituted for the equations type
variables to make the equations valid.
For example, suppose we have the equations
\begin{align*}
  `\tau_1 \to `\tau_2 &= `\tau_1 \to (`\tau_1 \to `\tau_3)\\
  `\tau_3 &= `\tau_4 \to `\tau_1\\
  `\tau_4 &= `\tau_4
\end{align*}
Applying the unification algorithm would give
\begin{align*}
  `\tau_2 &= `\tau_1 \to `\tau_4 \to `\tau_1\\
  `\tau_3 &= `\tau_4 \to `\tau_1
\end{align*}
A couple things to note are that
\begin{enumerate}
\item Obvious equalities (i.e. $x = x$) are removed.
  
\item All type variables are fully expanded on right hand side of the
  equations.
  This is why $`\tau_2 = `\tau_1 \to `\tau_4 \to `\tau_1$ and not
  $`\tau_2 = `\tau_1 \to `\tau_3$
\end{enumerate}
For a more detailed explanation of the unification algorithm, please
refer to \cite{Martelli1982AnEU}.

Now I will describe the type inference algorithm.
I will be omitting some details for the sake of clarity, see appendix
\ref{sec:static_semantics} for the complete formalization.
At all stages of type inference we have a context $\Gamma$, which is
simply the mapping of variables to the types of the values they are
bound to.
There are several cases
\begin{itemize}
\item If we have a variable $v$ then we take $t = \Gamma(v)$.
  If $t$ is a quantified type, then we apply it to free type
  variables.

\item If we have a lambda abstraction $\lambda v. e$, then we first
  declare a new unique type variable $`\tau$;
  infer the type $t$ of $e$ while setting $\Gamma(v) = `\tau$; and
  deduce the abstraction type as $`\tau \to t$.

\item If we have an application $e_1~e_2$ we first infer $e_1 : t_1$
  and $e_2 : t_2$; then let $`\tau$ be a new unique type variable;
  unify $t_1$ with $t_2 \to `\tau$; and deduce that $e_1~e_2$ has the
  type of whatever $`\tau$ unifies as.

\item If we have a let binding $(\thmlet{v := e_1;}{e_2})$, then we
  deduce the type of $e_1$ and quantify over it to get the type $t_1$;
  and infer the type $t_2$ of $e_2$ while setting $\Gamma(v) = t_1$;
  with the inference giving $(\thmlet{v := e_1;}{e_2}) : t_2$.
  Note that quantifying over a type $t$ means we take all the free
  type variable $`\tau_1,\ldots,`\tau_n$ in $t$, and return the type
  $\forall `\tau_1. \ldots \forall `\tau_n. t$.
  That is, we convert it into a quantified type.

\item For the fix-point operator we simply let $`\tau$ be a new type
  variable, and infer that $\thmfix : (`\tau \to `\tau) \to `\tau$.
\end{itemize}
That was a fairly brief description of the Hindley-Milner type
inference algorithm.
For more details or clarifications please see the appendix and
Milner and Damas'
work\cite{MILNER1978,Damas1982,Damas1984}\footnote{Milner's work comes
from a mathematical perspective, and would be less useful in to learn
from in this context}.


\subsection{Algebraic Data Types}

As note earlier quantified types can not be passed through function
application.
These means that complex functions that use Church encodings may fail.
For example, the common definition of the predecessor function for
natural numbers is
\begin{align*}
Z~x~f &= x;\\
S~n~x~f &= f~(n~x~f);\\
pair~x~y~f &= f~x~y;\\
pred~n &= (n~(pair~Z~Z)~(\lambda p.~p~(\lambda x~y.~pair~y~(S~x))))~(\lambda x~\_.~x);
\end{align*}
However, running $pred~(S~(S~Z))$ will give a complex type error due
to the fact that we swap the pair $x$ and $y$ on each update, giving
a circular type.

To improve usability the base language is thus extended with algebraic
data types (ADTs).
These allow us to easily construct new types.
The syntax for ADTs is
\begin{align*}
  &\mathrm{type}~T~`\tau_1~\ldots~`\tau_m~[\\
  &\qquad C_1(t_{11}, \ldots, t_{1n_1})\\
  &\qquad \vdots\\
  &\qquad C_k(t_{k1}, \ldots, t_{kn_k})\\
  &]
\end{align*}
Where $T$ is the type name, $`\tau_1, \ldots, `\tau_m$ are its type
parameters, and $C_1,\ldots,C_k$ are its constructors.
Each constructor $C_i$ forms a new function, with
$$C_i : \forall `\tau_1~\ldots~`\tau_m.~t_{i1} \to \cdots \to t_{in_i} \to T~`\tau_1~\ldots~`\tau_m$$.
A new built-in case matching function $\thmcase{T}$ is also formed,
with type
\begin{align*}
  \thmcase{T} : \forall `\tau_1~\ldots~`\tau_m~`\rho.
  &\quad T~`\tau_1~\ldots~`\tau_m \to (t_{11} \to \cdots \to t_{1n_1} \to `\rho)\\
  &\qquad \to \cdots \to (t_{k1} \to \cdots \to t_{kn_k} \to `\rho) \to `\rho
\end{align*}.
If we consider a value $z = C_i~x_1~\ldots~x_{n_i}$, then case matching
evaluates $(\thmcase{T}~z~f_1~\ldots~f_k)$ to $f_i~x_1~\ldots~x_{n_i}$.
Essentially we give a function for each constructor of $T$, and then
apply it the correct function to the arguments of the constructor.

For example, lists can be defined as
\begin{align*}
  &\mathrm{type}~\mathrm{List}~`a~[\\
  &\qquad \mathrm{nil}\\
  &\qquad \mathrm{cons}(`a, \mathrm{List}~`a)\\
  &]
\end{align*}
Here we have
\begin{align*}
  \mathrm{nil} &: \forall `a.~\mathrm{List}~`a\\
  \mathrm{cons} &: \forall `a.~`a \to \mathrm{List}~`a \to \mathrm{List}~`a
\end{align*}
If $x : t$, then using these constructors we could get
$(\mathrm{cons}~x~\mathrm{nil}) : \mathrm{List}~t$.
We also have
$$\thmcase{\mathrm{List}} : \forall `a `\rho.~
\mathrm{List}~a \to `\rho \to (`a \to \mathrm{List}~`a \to `\rho) \to `\rho$$,
which for example evaluates for each constructor as
\begin{align*}
  \thmcase{\mathrm{List}}~\mathrm{nil}~x~f &\Rightarrow x\\
  \thmcase{\mathrm{List}}~(\mathrm{cons}~y~l)~x~f &\Rightarrow f~y~l
\end{align*}

One interesting thing to note about THM's ADT is that they are
flexible enough to also allow for bottom types to be defined.
A bottom type is defined as
$$\mathrm{type}~\mathrm{Bottom}~[~]$$.
The important thing to note is that there is no way to construct a
value with type bottom (aside from infinite recursion).
This can make it useful for abstractions where the user wants to state
that we should never get a value in some location.
This simply demonstrates the flexibility of THM's ADTs.


\section{Implementation and Use}

THM is implemented as an interpreter written in Haskell using the
Stack build tool\cite{stack}.
Basic usage is to give the interpreter a set of file with function and
type definitions to load, after which it will print the types of
loaded functions, and then start the REPL to evaluate expressions.
An example session is
\begin{verbatim}
sh > rlwrap stack run lib/prelude.thm
if : Bool -> 'g -> 'g -> 'g
if = <function>

not : Bool -> Bool
not = <function>

and : Bool -> Bool -> Bool
and = <function>

or : Bool -> Bool -> Bool
or = <function>

...

\> and true false
false : Bool
\> or true false
true : Bool
\> case[Bottom]
case[Bottom] : Bottom -> 'a
\end{verbatim}
Note that THM provides a set of library files, with lib/prelude.thm
containing basic definitions.
Other library files providing natural number, lists, and stateful
semantics are also provided.
The source code can be found at
\url{https://github.com/lambda-11235/thm}.


\section{Conclusion}

THM is a Hindley-Milner typed programming language designed for
educational use.
It's formal specification is small enough to be described in 5 pages
(see the appendix), and its source code is composed of 10 files and
less than 800 lines of code.
In fact, the type checking and evaluation components (the main
components of interest in teaching) are less than 400 lines of code.
Compared to implementations of SML or Haskell, this makes THM a viable
project to study for graduate students trying to learn how to
implement type inference, ADTs, and lazy evaluation.
If you, the reader, are a professor or teacher who is or will be
teaching type theory, consider using this project for instructional
use.
Pull requests to improve the code's approachability are also
welcome\footnote{\url{https://github.com/lambda-11235/thm/pulls}}.


\printbibliography


\pagebreak


\appendix

\section{Syntax}
\label{app:syntax}

\subsection{Abstract Syntax}

$x$ indicates a variable, $e$ is an expression, $C$ is a data
constructor, $`\tau$ indicates a type variable, $t$ is a type, $T$ is a
type constructor, and $`\alpha$ is a quantified type variable.

\begin{align}
  \mathrm{TypeDef} &:= \mathrm{type}~T~`\tau_1~\cdots~`\tau_m \sbr{
                     C_1\del{t_{11},\ldots,t_{1n_1}} 
                     \cdots
                     C_k\del{t_{k1},\ldots,t_{kn_k}}
                     } \label{eq:typedef}
\end{align}

\begin{align}
  t &:= t_1 \to t_2 ~|~ T~t_1 \ldots t_n ~|~ `\tau ~|~ \forall `\alpha. t
\end{align}

\begin{align}
  e &:= \mathrm{Var} ~|~ \mathrm{Cons} ~|~ \mathrm{Lambda} ~|~ \mathrm{App} ~|~ \mathrm{Let} ~|~ \mathrm{Fix} ~|~ \mathrm{Case}\\
  \mathrm{Var} &:= x\\
  \mathrm{Cons} &:= C\\
  \mathrm{Lambda} &:= \lambda x. e ~|~ \lambda \_. e\\
  \mathrm{App} &:= e_1~e_2\\
  \mathrm{Let} &:= \thmlet{x = e_1;}{e_2}\\
  \mathrm{Fix} &:= \thmfix\\
  \mathrm{Case} &:= \thmcase{T}
\end{align}




\subsection{EBNF Concrete Syntax}
\label{sec:concrete_syntax}

Comments begin with \#.
Symbols as any non-keyword strings that match the regex\\
\verb~[a-zA-Z][0-9a-zA-Z]*~.
A number is \verb~[0-9]+~.

\begin{align}
  \bs{tydef} &\beq \bstr{type}, \bs{symbol}, \cbr{\bs{tyvar}},
               \bstr{[}, \cbr{\bs{condef}}, \bstr{]} ;\\
  \bs{condef} &\beq \bs{symbol}, \sbr{\bstr{(}, \cbr{\bs{type}}, \bstr{)}} ;
\end{align}

\begin{align}
  \bs{type} &\beq \bs{tycon}, \sbr{\bstr{->}, \bs{type}} ;\\
  \bs{tycon} &\beq \bs{tyatom} ~|~ \bs{symbol}, \cbr{\bs{tyatom}} ;\\
  \bs{typatom} &\beq \bs{tyvar} ~|~ \bstr{(}, \bs{type}, \bstr{)} ;\\
  \bs{tyvar} &\beq \bstr{'}, \bs{symbol} ;
\end{align}

\begin{align}
  \bs{funcdef} &\beq \sbr{\bstr{fix}}, \bs{symbol}, \cbr{\bs{symbol}}, \bstr{=}, \bs{expr}, \bstr{;} ;\\
  \bs{expr} &\beq \bstr{let}, \bs{funcdef}, \cbr{\bs{funcdef}}, \bstr{in}, \bs{expr} ;\\
              &\qquad|~ (\bstr{\backslash} | \bstr{\lambda}), \bs{arg}, \cbr{\bs{arg}}, \bstr{.}, \bs{expr} ;\\
                 &\qquad|~ \bs{app}\\
  \bs{app} &\beq \bs{atom} ~|~ \bs{app}, \bs{atom} ;\\
  \bs{atom} &\beq \bstr{(}, \bs{expr}, \bstr{)} ~|~ \bs{symbol} ~|~ \bs{case} ;\\
  \bs{arg} &\beq \bs{symbol} ~|~ \bstr{\_}\\
  \bs{case} &\beq \bstr{case}, \bstr{[}, \bs{symbol}, \bstr{]} ;
\end{align}

Note that the EBNF above has some syntatic sugar to make programming
easier.
The rules to convert from the main language to the abstract one are
\begin{align}
  \lambda x_1 x_2 \cdots. e
  &\Rightarrow \lambda x_1. \lambda x_2. \lambda \cdots. e\\
  \thmlet{x_1 := e_1;~x_2 := e_2;~\ldots}{e}
  &\Rightarrow \thmlet{x_1 := e_1;}{\thmlet{x_2 := e_2;}{\ldots~\mathrm{in}~e}}\\
  \thmlet{f~x_1~x_2~\cdots := e_1;}{e_2}
  &\Rightarrow \thmlet{f := \lambda x_1 x_2 \cdots. e_1;}{e_2}\\
  \thmlet{\thmfix~f~x_1~x_2~\cdots := e_1;}{e_2}
  &\Rightarrow \thmlet{f := \thmfix~\del{\lambda f x_1 x_2 \cdots. e_1};}{e_2}
\end{align}


\section{Semantics}
\label{app:semantics}


\subsection{Type Definitions}
\label{sec:type_defs}

Suppose we have a type definition following the pattern given in
\eqref{eq:typedef}.
There are three semantic rules that need to be verified.
\begin{enumerate}
\item For any pair of type variables
  $a, b \in \cbr{`\tau_1,\ldots,`\tau_m}$ that $T$ parameterizes over
  $a \neq b$.

\item For any type $t$ used in one of the constructors
  $C_1,\ldots,C_k$ we require
  $\freevars{t} \subseteq \cbr{`\tau_1,\ldots,`\tau_m}$.

\item For any type $t$ used in one of the constructors
  $C_1,\ldots,C_k$ we require that the only type constructors used are
  $T$ and previously defined type constructors.
\end{enumerate}


\subsection{Static}
\label{sec:static_semantics}

The following is largely adapted from a description of algorithm W
from Wikipedia\cite{wikiAlgW}.
Notable typing rules that extend those from Wikipedia are those for
recursive let bindings and matches.

$\newtyvar$ creates a unique type variable.
$\unify{t_1}{t_2}$ denotes the unification substitution of $t_1$ and
$t_2$.
It does by adding $t_1 = t_2$ to the set of unification equations and
then simplifying.
$\inst{\sigma}$ takes a possibly quantified type (i.e. of the form
$\sigma = \forall `\alpha_1, \ldots, `\alpha_n. t$) and produces the
$t_u$ by replacing the
quantified type variables with new unique ones, that is
\begin{align*}
  `\tau_1,\ldots,`\tau_n &= \newtyvar\\
  t_u &= [`\tau_1/`\alpha_1, \ldots, `\tau_n/`\alpha_n] t
\end{align*}
$\gen{\Gamma}{t}$ does the opposite of $\inst{\sigma}$, that is it
quantifies over all the free type variables in $t$ that not also free
in $\Gamma$, or
\begin{align*}
  \cbr{`\alpha_1, \ldots, `\alpha_n} &= \freevars{t} - \freevars{\Gamma}\\
  \gen{\Gamma}{t} &= \forall `\alpha_1~\ldots~`\alpha_n, t
\end{align*}
The syntax $\Gamma \vdash e : t$ means that in the context
$\Gamma$, $e$ has type $t$.


\begin{description}
\item[Var]
  \begin{equation}
    \frac{x : \sigma \in \Gamma \qquad t = \inst{\sigma}}{\Gamma \vdash x : t}
  \end{equation}

\item[Cons] Suppose $C(t_1,\ldots,t_n)$ is a constructor for some type
  $T~`\tau_1\cdots`\tau_m$, then
  \begin{equation}
    \frac{}{\Gamma \vdash C : \forall `\tau_1 \cdots `\tau_m, t_1 \to \cdots \to t_n \to T~`\tau_1 \cdots `\tau_m}
  \end{equation}

\item[Lambda]
  \begin{gather}
    \frac{`\tau = \newtyvar \qquad \Gamma, x : `\tau \vdash e : t}{\Gamma \vdash \lambda x. e : `\tau \to t}\\
    \frac{`\tau = \newtyvar \qquad \Gamma \vdash e : t}{\Gamma \vdash \lambda \_. e : `\tau \to t}
  \end{gather}

\item[App]
  \begin{equation}
    \frac{\Gamma \vdash e_1 : t_1 \qquad \Gamma \vdash e_2 : t_2 \qquad `\tau = \newtyvar \qquad \unify{t_1}{t_2 \to `\tau}}{\Gamma \vdash e_1~e_2 : `\tau}
  \end{equation}

\item[Let]
  \begin{equation}
    \frac{\Gamma \vdash e_1 : t \qquad \Gamma, x : \gen{\Gamma}{t} \vdash e_2 : t`}{\Gamma \vdash \thmlet{x = e_1;}{e_2} : t`}
  \end{equation}

\item[Fix]
  \begin{equation}
    \frac{`\tau_1, `\tau_2 = \newtyvar}{\Gamma \vdash \thmfix : (`\tau_1 \to `\tau_2) \to (`\tau_1 \to `\tau_2)}
  \end{equation}

\item[Case] Suppose we have a type definition following the pattern
  given in \eqref{eq:typedef}.
  \begin{equation}
    \frac{`\rho = \newtyvar}{\Gamma \vdash \thmcase{T} : T~`\tau_1~\cdots~`\tau_m \to (t_{11} \to \cdots \to t_{1n_1} \to `\rho) \to \cdots \to (t_{k1} \to \cdots \to t_{kn_k} \to `\rho) \to `\rho}
  \end{equation}
\end{description}


\subsection{Dynamic}
\label{sec:dynamic_sem}

Let $\sigma$ be the context of variable bindings and their
environments.
$\sigma$ thus has the type
$$\Sigma = X \to E \times \Sigma$$, where $X$ is the
set of all variables, and $E$ is the set of all expressions.
The notation $\sigma' = [(e, s)/x]\sigma$ means
$$\sigma'(y) =
\begin{cases}
  (e, s) &\text{if } y = x\\
  \sigma(y) &\text{if } y \neq x\\
\end{cases}$$
Note that $(e, s) : E \times \Sigma$ is the thunk formed from the
expression $e$ in some environment $s$.

\begin{description}
\item[Var]
  \begin{equation}
\frac{\sigma(x) = (e, s)}{\env{x, \sigma} \to \env{e, s}}
\end{equation}

\item[Lambda] Lambda expressions are fully reduced.

\item[App] 
  \begin{gather}
    \frac{\env{e_1, \sigma} \to \env{e_1', \sigma}}{\env{e_1~e_2, \sigma} \to \env{e_1'~e_2, \sigma}}\\
    \frac{}{\env{(\lambda x. e_1)~e_2, \sigma} \to \env{e_1, [(e_2, \sigma)/x]\sigma}}\\
    \frac{}{\env{(\lambda \_. e_1)~e_2, \sigma} \to \env{e_1, \sigma}}
  \end{gather}

\item[Let]
  \begin{gather}
    \frac{\env{e_1, \sigma} \to \env{e_1', \sigma}}{\env{\thmlet{x = e_1;}{e_2}, \sigma} \to \env{\thmlet{x = e_1';}{e_2}, \sigma}}\\
    \frac{}{\env{\thmlet{x = e_1;}{e_2}, \sigma} \to \env{e_2, [(e_1, \sigma)/x]\sigma}}
  \end{gather}

\item[Fix]
  \begin{equation}
    \frac{}{\env{\thmfix~f, \sigma} \to \env{f~(\thmfix~f), \sigma}}
  \end{equation}

\item[Case] Suppose we have a type definition following the pattern
  given in \eqref{eq:typedef}.
  Let $C_i$ be the $i$th listed constructor, then
  \begin{gather}
    \frac{}{\env{\thmcase{T}~(C_i~x_1~\cdots~x_n)~f_1~\cdots~f_k, \sigma} \to \env{f_i~x_1~\cdots~x_n, \sigma}}\\
  \end{gather}
\end{description}


\end{document}
